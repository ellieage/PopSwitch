% !TEX root = ./jwProj.tex

This section is motivation
for the definition of the pop-switch planar algebra.
We start with a summary of the definition of the graph planar algebra,
first defined in \cite{JonesGPA}.

Throughout this section,
fix a simple graph $\Gamma$.
For Jones, all planar algebras are shaded,
and $\Gamma$ is required to be bipartite.
We will ignore this issue.

Let $\mu$ be a function from the vertices of $\Gamma$ to $\mathbf{R}_{>0}$.
We will define the graph planar algebra $\mathcal{P}$
corresponding to $(\Gamma, \mu)$.

For each $k > 0$,
let $\mathcal{P}_{2k}$ be the vector space of complex valued functions
on the set of loops of length $2k$ on $\Gamma$.

Suppose $T$ is a tangle.
For each input disk of $T$,
let $v_b$ be a corresponding input vector.
We must define a corresponding output vector $v$.
Thus we must define
$v(\gamma)$ for every loop $\gamma$ in $\Gamma$
that has length equal to the number of endpoints on the outer boundary of $T$.

A {\em state} $\sigma$ of $T$ is a function
from the set of regions of $T$ to the set of vertices of $\Gamma$
such that adjacent regions are sent to adjacent vertices.

Suppose $r$ is a region of $T$.
This is a planar surface with boundary
that may include some right-angled corners.
The {\em Euler measure} $e(r)$
is defined in a similar way to the Euler characteristic,
using the usual formula $V - E + F$ for a triangulation of $r$.
The difference is,
every corner must be a vertex and only counts as $\frac{1}{4}$,
any other vertex on a boundary only counts as $\frac{1}{2}$,
and every edge on a boundary only counts as $\frac{1}{2}$.

We are finally ready to define the image vector $v$
of the vectors $v_b$
under the action of the tangle $T$.
$$
v(\gamma) =
\sum_\sigma
\left( \prod_r \mu(\sigma(r))^{e(r)} \right)
\left( \prod_b v_b(\sigma|_{\partial b}) \right).
$$
The sum is over all states $\sigma$ that are compatible with $\gamma$.
The first product is over all regions $r$ of $T$.
The second product is over all input disks $b$ of $T$.

The Temperley-Lieb planar algebra
is a subfactor planar algebra of type $A_\infty$.
It can be found inside
the graph planar algebra associated to $\Gamma = A_\infty$,
which is the ray with vertices indexed by positive integers.
The function $\mu$
assigns the quantum integer $[n]$ to the $n$th vertex.
(Note we are still assuming $q$ is not a root of unity.
If $q$ is a primitive $(n+1)$th root of unity
then we should use the graph $A_n$.)

Suppose $T$ is an oriented tangle.
Define a state of $T$
to be a function from the set of regions of $T$
to the set of vertices of $A_\infty$
such that,
for any strand of $T$,
if the region to its right is sent to vertex $n$
then the region to its left is sent to vertex $n+1$.
Thus,
a state is determined by the vertex associated to a single region.
In a sense,
the orientation on the strands
removes the ambiguity in the state of a Temperley-Lieb diagram.

Now suppose $T$ and $T'$ differ by a pop-switch relation.
There is an obvious correspondence between
states of $T$ and states off $T'$.
Furthermore,
the total Euler measure of the region associated to any given vertex is the same.
We therefore have
a well-defined embedding
of the pop-switch planar algebra
in the graph planar algebra of the graph $A_\infty$.

One can think of the pop-switch planar algebra
as a diagrammatic way to keep track of
computations inside the graph planar algebra of $A_\infty$.