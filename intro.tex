% !TEX root = ./jwProj.tex

The Temperley-Lieb algebras were first introduced by Temperley and Lieb \cite{TL}
in their work on transfer matrices in statistical mechanics.  
Vaughn F. R. Jones independently rediscovered Temperley-Lieb algebras
in his work on von Neumann algebras \cite{Jones}.
He assembled these algebras together to form the Temperley-Lieb planar algebra,
the simplest example of a subfactor planar algebra. 

%The $n$th Jones-Wenzl idempotent,first introduced in \cite{Wenzl},is a unique element of the $n$th Temperley-Lieb algebra.
%There is a unique Jones-Wenzl idempotent for each Temperley-Lieb algebra.
The Jones-Wenzl idempotents, first introduced in \cite{Wenzl}, are elements of the Temperley-Lieb algebras.
One way they arise naturally is in representation theory.
The Temperley-Lieb algebras encode
the category of representations of $U_q(\mathfrak{sl}_2)$,
and the Jones-Wenzl idempotents represent the irreducible representations.
Chapters in books have been devoted to them \cite{Kauffman}.
They have been categorified by \cite{JWcateg} and \cite{JWcateg2},
and generalized \cite{GenJW}.

While important, the Jones-Wenzl idempotents are difficult to write down explicitly.
The $n$th Jones-Wenzl idempotent  is a linear combination of
every diagram with $n$ non-intersecting strands.
The number of these diagrams is the $n$th Catalan number.
To find the coefficient of a given diagram requires a complicated algorithm
originally given by Frankel and Khovanov \cite{frenkel1997}
and later written down by Morrison \cite{Morr}.

In this paper,
we define the pop-switch planar algebra,
a new planar algebra that contains the Temperley-Lieb planar algebra.
Our original motivation was
a diagrammatic treatment of the graph planar algebra
introduced by Jones \cite{JonesGPA}. 
The pop-switch planar algebra captures  with simple diagrams the complicated calculations involved in working with objects in the graph planar algebra.
% The main result of this paper is that,
% in the pop-switch planar algebra,
% the Jones-Wenzl idempotent $p_n$
% is isomorphic to a direct sum of $n+1$ very simple diagrams.

The main theorem of this paper shows that each Jones-Wenzl idempotent is isomorphic to a direct sum of diagrams with only vertical strands.  
It is to be hoped that
this makes them easier to work with,
and gives a new approach to some open problems.
%\marginpar{eg?}



