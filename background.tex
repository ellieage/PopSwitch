% !TEX root = ../jwProj.tex

For convenience, we work over the field $\mathbb{C}$
and let $q$ be a nonzero complex number that is not a root of unity.
Many of the results hold over other fields,
but if $q$ is a root of unity
the proofs fail due to division by zero.

\begin{defn}
The $n$th {\em quantum number} is defined as 
$$[n] = [n]_q = \frac{q^n-q^{-n}}{q-q^{-1}}$$
and the {\em quantum binomial} is defined as
$$\qbinom{n}{k}=\frac{[n][n-1]\cdots[n-k+1]}{[k][k-1]\cdots[1]}$$
where $0 \le k \le n$ are natural numbers.
\end{defn}

We have the following identities.

\begin{lem}
\label{lem:q}
$[k+l]=[k][l+1] - [k-1][l]$.
\end{lem}

\begin{proof}
This follows from the definition and a simple computation.
\end{proof}

\begin{cor}
\label{cor:q}
$\qbinom{k+l}{l}=[l+1]\qbinom{k+l-1}{l}-[k-1]\qbinom{k+l-1}{l-1}$.
\end{cor}

\begin{proof}
After taking a common denominator and cancelling common terms,
this reduces to the previous lemma.
\end{proof}

\subsection{Planar algebras}

We won't define planar algebras in great detail.
See Jones' original paper \cite{Jones} for a formal definition.
See \cite{D2n} for a helpful introduction.

We will use what are sometimes called {\em vanilla} planar algebras.
These lack any of the optional extra features or properties
that are often included in the definition.

A planar tangle $T$ consists of:
\begin{itemize}
\item a disk $D$ called the {\em output disk},
\item a finite set of disjoint disks $D_i$ called the {\em input disks}
      in the interior of $D$,
\item a point called a {\em basepoint}
      of $\partial D$ and of each $\partial D_i$, and
\item a collection of disjoint curves called {\em strands} in $D$.
\end{itemize}
The strands can be closed curves,
or can have endpoints on $\partial D$
or $\partial D_i$ or both.
Apart from the endpoints,
the strands lie in the interior of $D$
and do not intersect $D_i$.
The basepoints do not coincide with endpoints of strands.
Planar tangles are considered up to isotopy in the plane.

It is sometimes possible to
insert a planar tangle $T_1$
into one of the input disks of another planar tangle $T_2$
to obtain a new planar tangle.
Specifically,
this is possible if
the number of endpoints on the output disk of $T_1$
is the same as the number of endpoints
on the chosen input disk of $T_2$.
Then we can use an isotopy to make the endpoints match up.
This still leaves an ambiguity of how to rotate $T_1$.
The basepoints remove this ambiguity:
we require the basepoint of the output disk of $T_1$
to coincide with the basepoint of the chosen input disk of $T_2$.

The planar tangles,
together with this operation of inserting one planar tangle
into an input disk of another,
form a rather general type of algebraic gadget called an
{\em operad}.
Briefly,
a planar algebra
is a representation of the operad of planar tangles.

More concretely,
a planar algebra $\mathcal{P}$ is a sequence of vector spaces 
$\mathcal{P}_i$ for $i \ge 0$.
Suppose $T$ is a planar tangle
with input disks $D_1, \dots, D_n$.
Let $d_i$ be the number of endpoints on $\partial D_i$
and let $d$ be the number of endpoints on $\partial D$.
Suppose $v_i \in \mathcal{P}_{d_i}$ for all $i$.
Then there is an action of $T$
$$T(v_1, \dots, v_n) \in \mathcal{P}_d.$$
The action of planar tangles
must be multilinear,
and it must be compatible with the operad structure
in a natural sense.

The definition of a planar algebra may seem complicated.
However it formalizes a fairly simple idea,
familiar to knot theorists,
of tangle-like diagrams
that can be glued together in arbitrary planar ways.
Perhaps the main novelty
is that we allow
formal linear combinations of diagrams,
which glue together in a multilinear way.

An example might help.

\subsection{The Temperley-Lieb planar algebra}

The simplest planar algebra is
the Temperley-Lieb planar algebra $\mathcal{TL}$.
The vector space $\mathcal{TL}_i$
is spanned by tangle diagrams that have no input disks
and $i$ endpoints on the output disk.

There is one relation.
A closed loop in a diagram
may be deleted
at the expense of multiplying by the scalar $q + q^{-1}$.
We call this the {\em bubble-bursting relation}.

If $i$ is odd then $\mathcal{T}_i$ is zero.
A basis for $\mathcal{T}_{2n}$
is given by tangle diagrams
that have $n$ strands and no closed loops.

In practice,
most planar algebras
can be thought of as formal linear combinations of diagrams
that are similar to Temperley-Lieb diagrams,
but with optional extra features,
like crossings, orientations, colors, or vertices.

\subsection{The category corresponding to a planar algebra}

Suppose $\mathcal{P}$ is a planar algebra.
We now describe how $\mathcal{P}$ can be thought of as a category.
In this context,
the input and output disks in the definition of $\mathcal{P}$
should be thought of as rectangles instead of round disks.

The category $\mathbf{C}$ corresponding to $\mathcal{P}$ is as follows.
\begin{itemize}
\item The objects are the non-negative integers.
\item The morphisms from $i$ to $j$
      are the elements of $\mathcal{P}_{i+j}$,
      thought of as having $i$ endpoints on the bottom of the rectangle
      and $j$ on the top.
\item The composition $f \circ g$
      is given by stacking $f$ on top of $g$.
\end{itemize}

Let $\mathcal{P}^j_i$ denote $\mathcal{P}_{i+j}$
with the elements treated as morphisms from $i$ to $j$.

An idempotent
is an element $p$ of $\mathcal{P}^n_n$
such that $p^2 = p$.


We can expand the objects in the category
by a construction known as the {\em Karoubi envelope}.
This new category $\mathbf{C}'$ is defined as follows.
\begin{itemize}
\item The objects of $\mathbf{C}'$ are
      the idempotents of $\mathbf{C}$.
\item The morphisms from $p$ to $q$
      are morphisms in $\mathbf{C}$ of the form $qxp$.
\end{itemize}

Next, note that
$\mathbf{C}$ and $\mathbf{C}'$
are also tensor categories,
where $x \otimes y$
is obtained by placing $x$ to the left of $y$.

Finally,
we can define a {\bf matrix category} of $\mathbf{C}'$.
The objects are formal direct sums of objects of $\mathbf{C}'$
and the morphisms are formal matrices.
Instead of this abstract definition,
all we need is the following lemma.

\begin{lem} \label{lem:dirSum}
Suppose $p$ and $q_1,\dots,q_n$ are idempotents such that
$$p = q_1 + \dots + q_n,$$
and $q_i q_j = 0$ whenever $i \neq j$.
Then 
$$p \simeq q_1 \oplus \dots \oplus q_n.$$
\end{lem}

\subsection{Jones-Wenzl idempotents}

The Jones-Wenzl idempotent $p_n$
is the unique element of $\mathcal{TL}^n_n$
such that
\begin{itemize}
\item $p_n \neq 0$
\item $p_n^2 = p_n$
\item $a p_n = 0$ if $a$ is any diagram
      that includes a strand with both endpoints at the bottom of the rectangle.
\item $p_n b = 0$ if $b$ is any diagram
      that includes a strand with both endpoints at the top of the rectangle.
\end{itemize}

% \begin{nota}
% Let $\input{intro/diagrams/pn.tex}$ $= p_n$.
% \end{nota}

Because of these last two properties, the Jones-Wenzl idempotents are sometimes referred to as ``uncappable."
If $q$ is a root of unity,
the Jones-Wenzl idempotents do not exist for all $n$.
%http://mathoverflow.net/questions/138213/when-are-jones-wenzl-projectors-defined


% vvvvvvvvvvvvvvv MOVED vvvvvvvvvvvvvvvvvv

%\subsection{The pop-switch planar algebra}

%\begin{defn}\label{def:pspa}
%Let the pop-switch planar algebra $\mathcal{PSPA}$ be the planar algebra generated by oriented strands modulo the following relations.
%\begin{enumerate}

%\item The pop-switch relations
%$$\begin{tikzpicture}[mystyle] 

%\draw [thin] (-1.1,-.9) rectangle (1.1,.9);


%\draw[
%        decoration={markings, mark=at position 0.125 with {\arrow{<}}},
%        postaction={decorate}
%        ]
%        (-.65,0) circle (2mm);

\circles{-.65}{0}{}{<}

\draw[
        decoration={markings, mark=at position 1.0 with {\arrow{>}}},
        postaction={decorate}
        ]
        (-0.1,-0.9) --(-0.1,0.9); 
\draw[
        decoration={markings, mark=at position 1.0 with {\arrow{>}}},
        postaction={decorate}
        ]
        (0.4,0.9) --(0.4,-0.9); 

\end{tikzpicture}
 = \begin{tikzpicture}[mystyle] 

%\draw[
%        decoration={markings, mark=at position 0.5 with {\arrow{<}}},
%        postaction={decorate}
%        ]
%         (-0.7,0.9) arc (180:360:6mm); 
%         
%\draw[
%        decoration={markings, mark=at position 0.5 with {\arrow{>}}},
%        postaction={decorate}
%        ]
%         (-0.7,-0.9) arc (180:0:6mm);         

\arc{.9}{.4}{0}{}{>}
\arc{-.9}{-.4}{0}{}{<}


\end{tikzpicture}
, \quad
%\input{intro/diagrams/ps1.tex} = \input{intro/diagrams/ps2.tex}$$

%\item The bubble-bursting relation
%$$\raisebox{-4pt}{\tikz \draw[->] (0,0) +(90:0.2) arc(90:460:0.2);} +
%  \raisebox{-4pt}{\tikz \draw[->] (0,0) +(90:0.2) arc(460:90:0.2);} =
%  (q + q^{-1}) \epsilon,$$
%where $\epsilon$ denotes the empty diagram.
%\end{enumerate}
%\end{defn}

%\begin{mainthm}
%\label{thm:main}
%The $n$th Jones-Wenzl idempotent is isomorphic to a direct sum of $n+1$ diagrams:
%$$p_n \simeq
%\raisebox{-0.5cm}{\tikz \draw[->](0,0) -- (0,1) node[inner sep=0.1,below right]{$n$};}
%\oplus
%\raisebox{-0.5cm}{\begin{tikzpicture} 
%    \draw[->](0,1) -- (0,0);
%    \draw[->](0.4,0) -- (0.4,1) node[inner sep=0.3,below right]{$n$};
%    \draw[->](0.8,1) -- (0.8,0);
%    \end{tikzpicture}}
%\oplus
%\dots
%\oplus
%\raisebox{-0.5cm}{\begin{tikzpicture}
%    \draw[->](0,1) -- (0,0) node[inner sep=0.3,above left]{$n$};
%    \draw[->](0.4,0) -- (0.4,1) node[inner sep=0.3,below left]{$n$};
%    \draw[->](0.8,1) -- (0.8,0) node[inner sep=0.1,above right]{$n$};
%    \end{tikzpicture}}
%$$
%\end{mainthm}

% ^^^^^^^^^^^^^^^^^^^^^^^^^^^^^^^^^^^^^^^^^^^^^^

%Denote the right hand side as $q_n$.  The isomorphisms will be $f\in \text{Hom} (p_n,q_n)$ and $g\in \text{Hom}(q_n,p_n)$.  To prove $f$ and $g$ are isomorphisms, we must show $f\circ g=q_n=id\in \text{Hom}(q_n,q_n)$ and $g\circ f=p_n=id\in \text{Hom}(p_n,p_n)$

%We break $p_n$ into $n+1$ smaller sums, matching each with the appropriate summand of $q_n$.
%We break $p_n$ into $n+1$ smaller sums, and show each is isomorphic to one of the $n+1$ summands in $q_n$.  
%Then we can put all $n+1$ isomorphisms into a matrix to give the needed isomorphism between $p_n$ and $q_n$.  We break up $p_n$ in such a way that non-matching terms compose to zero by exploiting the fact that diagrams with non-matching orientations on strands yield zero.  Thus ensuring $f\circ g$ is a direct sum.

%We use the fact that $p_n$ is an idempotent, $p_n=p_n^2=p_n id_n p_n$.  The $id_n$ is a sum of $2^n$ different ways of orienting $n$ vertical strands.  Break this sum into $n+1$ sums depending on whether $0, 1, 2, \ldots, n$ strands are oriented up.

%\begin{nota}
%Let $p^k_l$
%$$\begin{tikzpicture}[mystyle] 
%\draw [] (-.6,-.3) rectangle (.6,-.4);
%\draw [] (-.6,.3) rectangle (.6,0.4);
%\draw (0.0,0.0) node (k) {$k,l$};
%\end{tikzpicture}$$
%denote the sum of $\binom{k+l}{l}$ diagrams
%obtained from $p_{k+l} \mathrm{id}_{k+l} p_{k+l}$
%by orienting $k$ of the strands in the identity $\mathrm{id}_{k+l}$ up
%and the remaining $l$ strands down.
%\end{nota}

%\begin{lem}
%$p_n \simeq p^0_n \oplus \dots \oplus p^n_0$.
%\end{lem}

%\begin{proof}
%First,
%each $p^k_l$ is an idempotent.

%Next,
%$p_n = p_n \mathrm{id}_n p_n$.
%This is equal to the sum of $2^n$ diagrams
%obtained by orienting the $n$ vertical edges in $\mathrm{id}_n$.
%Thus
%$p_n = p^0_n + \dots + p^n_0$.

%Finally,
%they're ``orthogonal'' idempotents.
%\end{proof}


%$p_n=$ 
%\begin{tikzpicture}[mystyle] 
%\draw [] (-.6,-.3) rectangle (.6,-.4);
%\draw [] (-.6,.3) rectangle (.6,0.4);
%\draw (0.0,0.0) node (k) {$n,0$};
%\end{tikzpicture}
%$+$
%\begin{tikzpicture}[mystyle] 
%\draw [] (-.6,-.3) rectangle (.6,-.4);
%\draw [] (-.6,.3) rectangle (.6,0.4);
%\draw (0.0,0.0) node (k) {$n-1,1$};
%\end{tikzpicture}
%$+$
%\begin{tikzpicture}[mystyle] 
%\draw [] (-.6,-.3) rectangle (.6,-.4);
%\draw [] (-.6,.3) rectangle (.6,0.4);
%\draw (0.0,0.0) node (k) {$n-2,2$};
%\end{tikzpicture}
%$+\cdots+$
%\begin{tikzpicture}[mystyle] 
%\draw [] (-.6,-.3) rectangle (.6,-.4);
%\draw [] (-.6,.3) rectangle (.6,0.4);
%\draw (0.0,0.0) node (k) {$1,n-1$};
%\end{tikzpicture}
%$+$
%\begin{tikzpicture}[mystyle] 
%\draw [] (-.6,-.3) rectangle (.6,-.4);
%\draw [] (-.6,.3) rectangle (.6,0.4);
%\draw (0.0,0.0) node (k) {$n,0$};
%\end{tikzpicture}

%In Lemma \ref{lem:light} we match the $n+1$ sums in $p_n$ with the $n+1$ summands in $q_n$, showing each

%$
% \text{\input{mainThm/diagrams/kl.tex} } \cong
%    \begin{cases}
%      
%    \hspace{.3cm}\input{mainThm/diagrams/kl_iso_kbtl.tex} 

%, & \text{if}\ k\geq l \\[1.0cm]
%    \hspace{.3cm}\input{mainThm/diagrams/kl_iso_lbtk.tex} 

%, & \text{if}\ k< l 
%    \end{cases}
%$

%by $f_{k,l}$ and $g_{k,l}$.

%When we compose $f_{k,l}\circ g_{k,l}$, we have a Jones-Wenzl idempotent in the middle.  When expanding the Jones-Wenzl idempotent only one diagram is nonzero.  Thus we were able to avoid a large sum of diagrams.  The coefficient is found in Lemma \ref{lem:fish}. 

%Composing $g_{k,l}\circ f_{k,l}$ yields $(-1)^{kl} \qbinom{k+l}{l} X_{k,l}$ where 
 
% $
%    X_{k,l}=
%    \begin{cases}
      
%\scalebox{1.0}{\input{mainThm/diagrams/x_kbtl.tex}}

%, & \text{if}\ k\geq l \\[1.5cm]
%      \scalebox{1.0}{\input{mainThm/diagrams/x_lbtk.tex}}
%, & \text{if}\ k\leq l 
%    \end{cases}
%$.

%Theorem \ref{thm:pn} shows that $\begin{tikzpicture}[mystyle] 
%\draw [] (-.6,-.3) rectangle (.6,-.4);
%\draw [] (-.6,.3) rectangle (.6,0.4);
%\draw (0.0,0.0) node (k) {$k,l$};
%\end{tikzpicture}
%=(-1)^{kl} \qbinom{k+l}{l} X_{k,l}
%$ by induction on $k+l$.  

